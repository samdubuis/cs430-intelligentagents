\documentclass[11pt]{article}

\usepackage{amsmath}
\usepackage{textcomp}
\usepackage[top=0.8in, bottom=0.8in, left=0.8in, right=0.8in]{geometry}
% Add other packages here %


% Put your group number and names in the author field %
\title{\bf Excercise 4\\ Implementing a centralized agent}
\author{Group \textnumero : Student 1, Student 2}


% N.B.: The report should not be longer than 3 pages %


\begin{document}
\maketitle

\section{Solution Representation}

\subsection{Variables}
% Describe the variables used in your solution representation %

\subsection{Constraints}
% Describe the constraints in your solution representation %

\subsection{Objective function}
% Describe the function that you optimize %


\section{Stochastic optimization}

\subsection{Initial solution}
% Describe how you generate the initial solution %

\subsection{Generating neighbours}
% Describe how you generate neighbors %

\subsection{Stochastic optimization algorithm}
% Describe your stochastic optimization algorithm %


\section{Results}

\subsection{Experiment 1: Model parameters}
% if your model has parameters, perform an experiment and analyze the results for different parameter values %

\subsubsection{Setting}
% Describe the settings of your experiment: topology, task configuration, number of tasks, number of vehicles, etc. %
% and the parameters you are analyzing %

\subsubsection{Observations}
% Describe the experimental results and the conclusions you inferred from these results %

\subsection{Experiment 2: Different configurations}
% Run simulations for different configurations of the environment (i.e. different tasks and number of vehicles) %

\subsubsection{Setting}
% Describe the settings of your experiment: topology, task configuration, number of tasks, number of vehicles, etc. %

\subsubsection{Observations}
% Describe the experimental results and the conclusions you inferred from these results %
% Reflect on the fairness of the optimal plans. Observe that optimality requires some vehicles to do more work than others. %
% How does the complexity of your algorithm depend on the number of vehicles and various sizes of the task set? %

\end{document}